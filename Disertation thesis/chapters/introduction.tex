\chapter{Introduction} 

\section{Purpose and Research Question} 

This thesis focuses on creating and examining a system for generating code through speech commands using natural language processing (NLP) and machine learning (ML) technologies. The central research question that drives this work asks:

\begin{quote} 
What methods enable the combination of natural language processing and speech recognition to create correct and context-sensitive code from spoken commands?
\end{quote} 

The research intends to deliver a unique voice-driven programming prototype through its contributions which aim to enhance software development accessibility and efficiency.

\section{Approach and Methodology} 

Natural language processing (NLP) and machine learning (ML) applications have grown to include language translation along with content generation and conversational agents among other domains. This research creates a web-based platform which uses user speech to generate pertinent source code through applied software engineering technologies.


The methodology consists of:
\begin{itemize}
    \item Preprocessing speech input using automatic speech recognition (ASR)
    \item Converting spoken language into textual instructions
    \item Using trained NLP/ML models to interpret and convert the instructions into programming constructs
    \item Evaluating the generated code through testing and usability feedback
\end{itemize}

Python will be the main programming language, leveraging libraries such as TensorFlow, spaCy, and open-source speech-to-text APIs.

\section{Scope and Limitations}

To ensure the timely completion of this research, the project scope has been intentionally limited. The system will focus on generating code in a single programming language—Python—and will handle a defined subset of tasks such as function declarations, loops, and conditional statements. Complex multi-line logic, debugging suggestions, and real-time feedback will be outside the scope of this implementation.

Furthermore, the speech recognition system will be trained to understand English-language input and may not handle accents or multi-language commands effectively in its initial version. Despite these limitations, the project will provide a proof-of-concept implementation and testing environment for future expansion.

\section{Target Group}

The target group for a speech-based code generation system includes:
\begin{itemize}
    \item Beginners and non-programmers seeking an easier way to create software
    \item Individuals with disabilities that make typing difficult
    \item Developers interested in rapid prototyping through voice commands
    \item Educators and students in computer science exploring assistive learning tools
\end{itemize}

By lowering the entry barrier, this system will make programming accessible to all. 

\section{Outline}

The thesis will begin with a literature review of the state of the art in NLP and ML to be used in code generation tools. The review will also highlight the limitations and problems associated with these tools and outline areas of future research.

Following the literature review, the thesis will cover the methodology and deployment of the web application used as the user interface. This application allows users to record and play back verbal commands and retrieve the generated code, as well as the NLP and ML components that facilitate the speech-to-code functionality.

The implementation chapter will also cover system architecture, development process, and integration of significant technologies. The evaluation portion will cover speech-based system testing, including measures of usability as well as the quality of code generated.

The final chapters will cover the outcomes, present an extensive comparison with existing methodologies, and examine the limitations. The thesis will finalize with a summary of the conclusions, debating implications for the software industry, and recommending avenues for further research.

\section{Research Activity}
Beyond the core focus of my thesis, I have also maintained active engagement in the broader domain of Intelligent Systems, particularly with an emphasis on sustainability and renewable energy applications. This interdisciplinary exploration led to the conception and development of a project titled \textit{"Renewable Energy Investment Calculator"} \cite{9894203}, which was presented at the \textit{2022 International Conference on INnovations in Intelligent SysTems and Applications (INISTA) 2022}. This work represents a critical intersection of intelligent system design, environmental consciousness, and economic modeling.

The central objective of the project was to create a computational tool capable of simulating and evaluating the long-term financial outcomes of various renewable energy investments. The calculator provides users with an intuitive platform to input financial parameters and energy consumption data, enabling a comparison between traditional energy sources and renewable alternatives such as solar and wind energy. The system models cost-benefit projections over multiple time horizons, incorporating factors like installation costs, maintenance, government incentives, and energy prices. In doing so, it empowers individuals, policymakers, and businesses to make data-driven and environmentally responsible decisions regarding energy investments.

Our approach combined principles from artificial intelligence, economic modeling, and software engineering. From a technical standpoint, we employed intelligent algorithms to assess and forecast energy savings, ensuring the results adapt dynamically to user-provided inputs. The result was a decision-support system that contributes meaningfully to discussions around climate change mitigation, energy efficiency, and sustainable development. Furthermore, the project addresses global policy trends advocating for green transition and highlights how intelligent systems can facilitate these large-scale shifts.

Integrating this research activity into my academic portfolio demonstrates not only the practical utility of intelligent systems but also my capacity to engage with diverse and socially relevant problem domains. It highlights my ability to apply interdisciplinary knowledge—bridging technical design with societal impact—to develop tools that support real-world decision-making. Moreover, it underlines my commitment to addressing contemporary global challenges through innovation and responsible system design. This work stands as a testament to both the adaptability of intelligent systems and my ongoing pursuit of meaningful, problem-solving research.