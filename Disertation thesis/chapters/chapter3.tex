% \chapter{Results}

% This section p

% \section{Case Studies}

% To test the performance and effectiveness of ...The results can be seen in Listings \ref{lst:html}, 

% \begin{lstlisting}[basicstyle = \footnotesize, language=HTML, label={lst:html}, caption=HTML Generated Code]
% <!DOCTYPE html>
% <html>
%   <head>
%     <meta charset="utf-8">
%     <title>Coin Flip</title>
%     <link rel="stylesheet" href="style.css" />
%   </head>
%   <body>
%     <div class="container">
%       <div id="result"></div>
%     <button class="btn" id="flipCoinBtn">Flip Coin</button>
%     <script src="script.js"></script>
% </body>
% </html>
% \end{lstlisting}


% \section{Automated vs Human Evaluation}

% While automated evaluation metrics ....
% \begin{center}
% \begin{table}
%     \centering
%     \rowcolors{1}{}{gray!10}
%     \begin{tabular}{ |p{7.5cm}|p{7.5cm}| } 
%     \hline
%     ChatGPT & Our Tool \\
%     \hline
%     Trained on a large dataset & Trained on limited data  \\ 
%     Sometimes only provides explanations for tasks and does not ganerate any code & Always generates some code \\ 
%     Wider range of domain knowledge & Limited domain knowledge \\ 
%     May not capture the entire underlying context & May not capture the entire underlying context \\
%     May generate the code in a single file & Always will separate the logic in separate files\\
%     Does not use semantic tags (many times it uses the $<$div$>$ element by default) & Also uses semantic tags such as $<$fieldset$>$\\
%     Does not adapt so easily to new syntax patterns & Adapts easily to new syntax patterns\\
%     Needs big amounts of data for fine-tuning & Achieves good performance with relatively smaller amounts of fine-tuning data\\
%     May not generate functional code (example of an analog clock generated by ChatGPT) & May not generate functional code\\
%     \hline
%     \end{tabular}
%     \caption{ChatGPT vs Our Tool}
%     \label{tab:comparison}
% \end{table}
% \end{center}

% \section{Usability Testing}

% To further evaluate the usability of the ...

% \subsection{Evaluation Procedure}

% Each participant was given ...

% \subsubsection{Feedback and Observations}

% Upon completion of the evaluation sessions, participants were asked to provide their feedback on various aspects of the tool. Their responses were analyzed to identify common problems and patterns. The following sections summarize the main findings:

% \subsubsection{Effectiveness}

% The participants generally found the ...

% \subsubsection{Usability}

% In terms of usability, the tool received positive feedback from the participants. They appreciated ....

% \subsubsection{Limitations and Suggestions for Improvement}

% Despite the overall positive feedback, a few limitations and areas for improvement were identified by the participants. 

% \subsection{Evaluation Conclusion}

% In conclusion, the evaluation by our colleagues demonstrated