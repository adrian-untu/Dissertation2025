\chapter*{Motivation}

Software application creation serves as a fundamental building block for today's digital environment. From personal tools to large-scale enterprise systems, software drives innovation across all sectors—including healthcare, finance, education, and autonomous technologies. As the reliance on digital tools increases, so does the need for more accessible and efficient ways of developing them.

Traditionally, creating software code requires technical proficiency, domain knowledge, and an understanding of programming syntax and structure. This creates a barrier for non-technical users, limits collaborative innovation, and contributes to increasing development costs. Additionally, this complexity restricts individuals with disabilities or without formal training from participating in software creation, reinforcing the digital divide.

Recent advancements in natural language processing (NLP), automatic speech recognition (ASR), and machine learning (ML) offer promising opportunities to address these challenges. By translating spoken natural language into executable source code, intelligent systems can significantly lower the threshold for software development. This democratization of programming enables users to express logic verbally and generate syntactically correct and context-aware code through a user-friendly interface.

However, the relevance of this thesis extends beyond software accessibility and user empowerment. In parallel with the rise of intelligent systems is the growing demand for ethical transparency in autonomous decision-making, especially in areas like self-driving vehicles, healthcare diagnostics, and AI policy enforcement. A key concern is how systems interpret and respond to ethical dilemmas, particularly when they involve conflicting outcomes.

A major obstacle in studying such dilemmas lies in the diversity of ethical frameworks and contextual data used in decision-making. Therefore, a crucial part of this research is not only generating systems through speech but also building a prototype capable of fetching, comparing, and analyzing ethical data across databases—to surface consistent patterns, contradictions, or edge-case scenarios. The system is intended to support reasoned, data-informed decision-making in ethically charged contexts.

For example, when evaluating moral scenarios like those found in autonomous vehicle decision-making (e.g., the Moral Machine dataset), intelligent systems can compare real user responses and ethical choices across diverse populations and scenarios. Such systems can aid researchers, engineers, and policymakers by offering interpretable visual and logical outputs that reflect ethical trade-offs and highlight possible standards or inconsistencies.

This thesis is motivated by a convergence of goals:

\begin{itemize}
\item Making software creation more accessible through voice-driven systems and speech-to-code translation
\item Empowering individuals with limited technical background or physical ability to participate in programming and automation
\item Creating intelligent systems capable of analyzing complex ethical data, drawing insights from cross-database comparisons, and assisting in automated ethical reasoning
\item Offering tools for transparency and informed decision-making in the design of autonomous systems
\end{itemize}

The research embraces a multidisciplinary vision, combining software engineering, human-computer interaction, natural language understanding, and applied ethics. By integrating these domains, the resulting platform demonstrates how intelligent systems can serve not only as productivity tools but also as facilitators of socially and morally aware automation.

In summary, this work contributes to a future where inclusive design, ethical reasoning, and intelligent automation converge—reshaping both how we program and how we confront the ethical implications of machine decisions.
\addcontentsline{toc}{chapter}{Motivation}