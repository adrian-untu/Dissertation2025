undefined
\chapter{Introduction}

\section{Motivation}

The development of software applications is a cornerstone of the modern digital world. However, writing code can be complex and time-consuming, requiring deep domain knowledge and technical expertise. This creates barriers for non-technical users and increases development costs for businesses. As software becomes increasingly integrated into daily life and industry, there is a growing demand for more intuitive, accessible, and efficient ways to generate code. One such promising approach is the use of natural language input—particularly speech—to generate code automatically using intelligent systems.

\section{Purpose and Research Question}

This thesis aims to explore and implement a speech-based code generation system using natural language processing (NLP) and machine learning (ML) technologies. The main research question guiding this work is:

\begin{quote}
\textit{How can natural language and speech recognition technologies be integrated to generate syntactically correct and contextually relevant code from user voice commands?}
\end{quote}

By answering this question, the research seeks to contribute a novel prototype that enables voice-driven programming, potentially improving accessibility and efficiency in software development.

\section{Approach and Methodology}

In recent years, the use of natural language processing (NLP) and machine learning (ML) has expanded across multiple domains, including language translation, content generation, and conversational agents. This thesis applies these technologies to software engineering by developing a web-based system that takes user speech as input and outputs relevant source code.

The methodology consists of:
\begin{itemize}
    \item Preprocessing speech input using automatic speech recognition (ASR)
    \item Converting spoken language into textual instructions
    \item Using trained NLP/ML models to interpret and convert the instructions into programming constructs
    \item Evaluating the generated code through testing and usability feedback
\end{itemize}

Python will be the main programming language, leveraging libraries such as TensorFlow, spaCy, and open-source speech-to-text APIs.

\section{Scope and Limitation}

To ensure the timely completion of this research, the project scope has been intentionally limited. The system will focus on generating code in a single programming language—Python—and will handle a defined subset of tasks such as function declarations, loops, and conditional statements. Complex multi-line logic, debugging suggestions, and real-time feedback will be outside the scope of this implementation.

Furthermore, the speech recognition system will be trained to understand English-language input and may not handle accents or multi-language commands effectively in its initial version. Despite these limitations, the project will provide a proof-of-concept implementation and testing environment for future expansion.

undefined